%% Los cap'itulos inician con \chapter{T'itulo}, estos aparecen numerados y
%% se incluyen en el 'indice general.
%%
%% Recuerda que aqu'i ya puedes escribir acentos como: 'a, 'e, 'i, etc.
%% La letra n con tilde es: 'n.

\chapter{State of the art}
\newpage

This chapter explains techniques and studies related to this work. The techniques explained in this chapter are related to \textit{Big Data}, information systems made with simple microcontrollers \cite{microcontrollers} and Internet of Things (IoT) projects.\\

\section{Big Data studies}

Big Data is a recent technology based on systems which require huge amount of information. The information produced in Big Data systems refers to sensors and machines. These sensors and machines automatically produce information which is stored in \textit{Information Systems}. The information Systems have resources to store huge amount of information, and also, can analyse the information stored to present results and predictions in \textbf{Real Time}.\\

There are studies which use this technology. Also, there are companies which use this technology as a business strategy. Some of them, are explained in this section:

\begin{itemize}

\item Google constantly develops new products and services that have Big Data algorithms. Google uses Big Data to refine its core search and ad-serving algorithms.

\item Autonomous driving. Companies involve in Autonomous Driving use Big Data to design algorithms. These algorithms recognize objects, people and also, take information from the different external factors that have influence on driving experience \cite{bmw_driving}.

\item Energy Future Holdings Corporation is an electric utility company. The majority of the company's power generation is through coal- and nuclear-power plants. The company used Big data to install smart meters. The smart meters allows the provider to read the meter once every 15 minutes rather than one month.

\item McLaren Racing Limited is a British Formula One team. The racing car team uses real-time car sensor data during car races, identifies issues with its racing cars using predictive analytics and takes corrective actions pro-actively before it is too late.

\item Verizon uses Big Data to enhance mobile advertising. A unique identifier is created when the user registers in the website. The identifier allows advertiser to use information from the desktop computer. Marketing messages can be delivered to you mobile phone using this information.

\end{itemize}

\section{Information Systems based on microcontrollers}

During last years, microcontrollers became more important on \textit{Information Systems}. They have important features and can offer control to different devices to get information. Obtaining information from systems is one of the most important epics in Information Systems.\\

There are several studies where microcontrollers are used in Information Systems. Most important studies are explained in this section:

\begin{itemize}

\item Smart Room Temperature Controller Atmega \cite{smart_room_controller}. This project is used as an alternative to control the temperature of a room. It measures the temperature of a room and notify the user when the threshold is reached.

\item Automatic water plants \cite{diy_blokefollow_watering_nodate}. The developer uses a microcontroller to control when it is needed to water plants. It has a designed circuit for its specific purpose.

\item An Automated Metrics System to Measure and Improve the Success of Laboratory Automation Implementation \cite{benn_automated_2006}. It is an application which collects information from a distributed system. The data has collected over 20 months from at least 28 different workstations. This system prints charts and generates analysis with the stored information.

\end{itemize}

\section{Relevance of soil moisture content}

Knowledge of soil–water content and soil hydraulic properties is essential for determination of flow of water and water balance transport of applied chemicals to plants and ground water, transport of dissolved salts and  pollutants through soil system irrigation management, and precision  maintenance of infrastructures. Thus, reliable results from numerical models of water flow and solute transport are critical for use by regulatory agencies. The accuracy of predictions is often limited by, among other things, the adequacy of hydraulic property estimations   Understanding such a variability is essential for a thorough comprehension of the processes it is related to.  Particularly, the analysis of temporal stability of soil moisture has proven to be useful in hydrological research at the watershed scale and in specific civil engineering infrastructures such as earth dams. 

Soil moisture is defined as the water contained in the unsaturated soil zone, also called vadose zone. In practice, often only a fraction of soil moisture is relevant or measureable. Thus, soil moisture needs to be considered with regard to a given soil volume.

\begin{equation}
    \theta = \frac{volume of water}{volume of soil}
\end{equation}

Another important quantity is the actual maximum soil moisture content of the given soil volume, i.e. the porosity. This maximum soil moisture content is defined as the saturation soil moisture content. Based on the definition of saturated soil moisture, we can established the term $\theta_s$ as a ratio of the actual soil moisture content.

\begin{equation}
    \theta_s= \frac{\theta}{\theta_{SAT}}
\end{equation}

For determining the soil moisture characteristics and parameters, a physical scale model of an earth embankments was built. Physical model for teaching and research purposes has been widely employed for under-graduate students due to gives a great effectiveness improving to unfamiliar with these conceptual understanding. It helps into to ease the assimilation  of  the  physic  concepts  around  the  flow  of  water  through  this  infrastructure,  i.e., permeability of soil, porosity, infiltration capacity and movement of water. 

Thanks to physic scale models,  the flux of water within the soil could be well understood with special attention to the slide of the dam when water spills over it or the influence of water erosion on the development of unstable failure mechanisms. In our case, the water flow movement through the soil following the voids of this material and the influence of the grain size on it is explained. 

\newpage
\newpage