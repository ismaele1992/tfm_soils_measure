%% Los cap'itulos inician con \chapter{T'itulo}, estos aparecen numerados y
%% se incluyen en el 'indice general.
%%
%% Recuerda que aqu'i ya puedes escribir acentos como: 'a, 'e, 'i, etc.
%% La letra n con tilde es: 'n.


\newpage


%\begin{abstract}
%blablabla
%\end{abstract}
\chapter*{}
\begin{center}
\textbf{Abstract}
\end{center}

This document includes a study of the behavior of different materials using an Information System based on using standard humidity sensors. The objetive of this study is determining if it is possible to use low cost solutions that could register information related with the behavior of different kind of floors, having variations in compaction processes and environmental factors.\\

It will be defined the Information System's arquitecture that has being designed with its different Software and Hardware components. The philosophy of this study is how it is possible to use different and newer technologies existing in Information Engineering on different studies made by other Engineerings.\\

%This document include a study of an array of nanoparticles based on electomagnetic analysis. The main goal will be %study what is the behavior of the array when we apply a perpendicular EM wave over the array with a wavelength %where the scattering will be maximum. Then we changes the size of the array to check if the modes in this problem %are dependent of it dimension.\\

%We use a simulation software with a cluster because we need reduce the proccess time due to the size of the problem.

\noindent
{\bf Keywords:\newline}
Information System, NoSQL, Big Data, Arduino, Raspberry Pi, Humidity Sensor, MongoCheff, 3TStudio, Python, Ubuntu Server, NodeJS, C, Software, Hardware, MongoDB

\thispagestyle{empty}
\newpage

\chapter*{}
\begin{center}
\textbf{Resumen}
\end{center}

Este documento recoge el estudio del comportamiento de diversos materiales mediante el uso de un Sistema de Información basado en la utilización de sensores de humedad. El objetivo a alcanzar en este estudio es determinar si es factible optar por soluciones de bajo coste que registren información del comportamiento de distintos sustratos ante variaciones en procesos de compactación y factores ambientales.\\

Se definirá la arquitectura del sistema de información que ha sido diseñado junto con sus respectivos componentes \textit{Hardware} y \textit{Software}. La filosofía de este trabajo busca extrapolar las nuevas tecnologías existentes en el campo de la Ingeniería Informática en estudios realizados por otras ingenierías.

%\thispagestyle{empty}
%\newpage

\thispagestyle{empty}

\newpage
