%% Los cap'itulos inician con \chapter{T'itulo}, estos aparecen numerados y
%% se incluyen en el 'indice general.
%%
%% Recuerda que aqu'i ya puedes escribir acentos como: 'a, 'e, 'i, etc.
%% La letra n con tilde es: 'n.


\newpage


%\begin{abstract}
%blablabla
%\end{abstract}
\chapter*{}
\begin{center}
\textbf{Abstract}
\end{center}

This document includes a study of the behavior of different materials using an Information System based on standard humidity sensors. The objetive of this study is determining if it is possible to use low cost and low consumptiom solutions that could register information related to the behavior of different kind of soils, having variations in compaction processes and environmental factors.\\

This document evaluates diverse software methodologies to apply to develop the purpose solution. All the methodologies includes information about their main features, considering if they are appropriate to satisfy the requirements of this project.\\

The architecture of the designed Information System and its Software and Hardware components are defined. The objective of this study is to apply and to analyze different and newer technologies existing in Information Engineering on different studies made by other Engineerings.\\

Arduino and Raspberry PI bring solutions to specific purpose systems. These microcontrollers are commonly applied to systems where sensors are required. On the other hand, automation is a requirement which is satified by this technology. This study includes a solution based on both microcontrollers.\\

Networking becomes relevant on Information Systems. The communication between the different devices which form part of a system is required. Therefore, a study of the network of sensors is required. This investigation proves the manner to interconnect devices with varied granularities. This knowledge is applicable to bigger systems.\\

The results obtained by the Information System are analyzed. An analysis is done to evaluate if the chosen solution is applicable to this study.\\

An annexed is included in this document. This annexed explains how to work with the developed Information System and it concludes with the required operations to maintain it. On the other hand, there is a section dedicated to explain most common issues.

%This document include a study of an array of nanoparticles based on electomagnetic analysis. The main goal will be %study what is the behavior of the array when we apply a perpendicular EM wave over the array with a wavelength %where the scattering will be maximum. Then we changes the size of the array to check if the modes in this problem %are dependent of it dimension.\\

%We use a simulation software with a cluster because we need reduce the proccess time due to the size of the problem.
\newpage
\noindent
{\bf Keywords:\newline}
Information System, NoSQL, Big Data, Arduino, Raspberry Pi, Humidity Sensor, MongoCheff, 3TStudio, Python, Ubuntu Server, NodeJS, C, Software, Hardware, MongoDB

\thispagestyle{empty}
\newpage

\chapter*{}
\begin{center}
\textbf{Resumen}
\end{center}

Este documento recoge el estudio del comportamiento de diversos materiales mediante el uso de un Sistema de Información, basado en la utilización de sensores de humedad. El objetivo a alcanzar en este estudio, es determinar si es factible optar por soluciones de bajo coste y bajo consumo que registren información del comportamiento de distintos sustratos ante variaciones en procesos de compactación y factores ambientales.\\

En este documento, se evalúan las distintas metodologías de desarrollo software aplicables para desarrollar la solución propuesta. En cada metodología de desarrollo software, se recogen principales características teniendo en cuenta si se adaptan a las necesidades requeridas en este proyecto.\\

Se define la arquitectura del sistema de información que ha sido diseñado junto con sus respectivos componentes \textit{Hardware} y \textit{Software}. La filosofía de este trabajo extrapola las nuevas tecnologías existentes en el campo de la Ingeniería Informática, a estudios realizados por otras ingenierías.\\

Arduino y Raspberry PI se presentan como microcontroladores como soluciones a sistemas de propósito específico. Su utilización se extiende principalmente a sistemas donde están presentes sensores. Por otro lado, esta tecnología satisface sistemas donde la automatización forma parte de un requisito más. Este estudio proporciona una solución en base a estos microcontroladores.\\

Las redes de comunicación destacan en Sistemas de Información. La comunicación entre dispositivos que forman parte de sistemas es un requerimiento que se considera. Como requisito adicional, se estudia la arquitectura relacionada con la red de sensores diseñada. Este trabajo expone cómo se interconectan dispositivos de diversas granularidades. Este conocimiento es extrapolable a complejos sistemas.\\

Se realiza un análisis de los resultados obtenidos mediante el Sistema de Información desarrollado. Con este estudio, se evalúa si la solución escogida es apropiada para el estudio propuesto.\\

Al final del documento, se adjunta un anexo con la información necesaria para trabajar con el Sistema de Información, junto con las distintas operaciones a desempeñar si fuese necesario realizar tareas de mantenimiento o resolución de problemas comunes en el propio sistema.

%\thispagestyle{empty}
%\newpage

\thispagestyle{empty}

\newpage
\newpage
