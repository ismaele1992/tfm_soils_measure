
\chapter{Materials and methods of study}
\newpage

This section of the memory explains the aspects related to the materials used to build the project. Additionally, it describes the methods used to obtain results and the technologies applied to develop the system.

\section{Materials}

To carry out this project, it was necessary to use numerous components supplied by the \textit{University of Extremadura}. These components are characterized by their diversity and their reduced cost. The prototype designed for the study uses free hardware components. Free hardware components have good features like their flexibility when building a system is required.\\

It was decided to used the aforementioned components since the manufacturer provides a lot of documentation. The documentation of these components allows to develop new features and provide flexibility and customization to the final user.\\

There are several components which form part of this study. These components are hardware devices and software features. Also, tools and frameworks are relevant in this project. All of them allow to develop the desired solution. This information is listed in this section.

\subsection{Arduino Mega}

It is more powerful than Arduino Uno, Arduino Nano and Arduino Micro. It has more available connections used as inputs and outputs. Thus, making possible to create more complex systems.\\

In this study we decided to use and Arduino Mega because it is a device which scales better than other models. Having a higher number of available connections allows the developer to extend easily the system.\\

\begin{figure}[H]
\begin{centering}
\includegraphics[scale=0.08]{IMGS/ARDUINO_MEGA.png}
\caption{Arduino Mega \label{Arduino_Mega}}
\end{centering}
\end{figure}

\textit{Arduino} is an embedded board which allows to program it using its framework, \textit{Arduino IDE}. With this framework it is possible to develop small and powerful programs which allow the control all the functions available in the system. The programs are developed in the C programming language using the \textit{Arduino IDE}. The \textit{Arduino IDE} includes many libraries and allows the developer to create his own libraries to customize the control of this board..\\

The most important capabilities of the \textit{Arduino Mega} are described in the following points:

\begin{itemize}

\item \textbf{\textit{Analogical outputs}}. These interfaces allow the programmer to read and control sensors. The read voltage of the sensors goes from the 0 value to 1023 value.

\item \textbf{\textit{Digital outputs}}. The modulation of the read signal uses binary values instead of curves to represent the information. Also, there are digital signal which convert the output values to analogical signals using the \textit{PWM} technology.

\item \textbf{\textit{Processor}}. This processor is a microcontroller board base on the ATmega2560. Its throughput reaches 16MIPS at 16MHz which is enough for embedded systems.

\item \textbf{\textit{Flash Memory}}. This memory is used to upload the developed software. Also, there is a section in this memory which is dedicated to the bootloader of the device. This bootloader allows the device to boot up.

\end{itemize}

\subsection{Raspberry Pi 3}

It is an ARM-based microcontroller. This device has a processor used in mobile devices like mobile phones, tablets and Chromecasts. It has a good performance and a low energy consumption. It allows the installation of light \textit{Unix} distributions.
	
\begin{figure}[H]
\begin{centering}
\includegraphics[scale=0.3]{IMGS/RPI_3.JPG}
\caption{Raspberry PI 3 \label{RPI_3}}
\end{centering}
\end{figure}

This project works with a \textit{Raspberry PI 3} using \textit{Raspbian Freezy} as operative system. This platform has a good support and it is proven to be stable. The community of Raspbian is strong and brings customized solutions for embedded systems.\\

\textit{Raspberry PI 3} is a device that has useful features e.g. storing information coming from sensors. It is able to manage a NoSQL database managing petitions and queries coming from users and developers.\\

The most outstanding features of the device are enumerated as follows:

\begin{itemize}

\item \textit{Mobility}. This device is small and easy to move to other places due to its wireless adapters.

\item \textit{Portability}. It is possible to install platforms in external storage and run the installed platform in every Raspberry PI 3.

\item \textit{Maintenance}. The maintenance of a Rasperry PI is low. It mainly involves the run of Software update. It is a device that does not have air cooling or other mechanical devices and it is possible to use it in industrial environments under harsh conditions.

\item \textit{Cost}. As aforementioned, the cost of the device and its components is low. For instance, its SD Card, power supply, network adapters or peripherals.

\end{itemize}

Knowing the capabilities of this device, in the following section it will be analysed the technical specifications of the \textit{Raspberry PI 3}:

\begin{itemize}

\item \textit{Processor}. It is based on ARM architecture. These processors have a low consumption power and good performance. Parallel and concurrent programs can be deployed such as a solution for complex systems.

\item \textit{Graphic processor}. This microcontroller allows to process images and textures. It is appropriated to develop solution where image processing is required.

\item \textit{Memory RAM}. It uses 1GB LPDDR2 memory which is just enough to run simple and advanced applications.

\item \textit{GPIO connections}. All these interfaces are used to connect peripheral and sensors. Sensors are programmed over these interfaces. It is common interface used in embedded systems.

\item \textit{SD slot}. This socket allows to connect a SD card where the operative system is deployed. It supports large and fast SD cards which are dedicated to the operative system and applications.

\item \textit{HDMI output}. Used to connect monitors and big screens.

\item \textit{Mini-Jack output}. Allows to connect the device to audio systems. Also, it is used to connect sensors which requires this connector.

\item \textit{USB ports}. These interfaces are used to connect external peripheral devices.

\item \textit{Ethernet interface}. It has a standard 1Gbit Ethernet via \textit{RJ45}.

\item \textit{Wi-Fi adapter}. It is integrated in the board and provides mobility to customized systems.

\end{itemize}

\subsection{Router}

It is a useful component which creates a network to allow the user to connect to the system and retrieve all information needed. There are some possibilities to initialize a connection with the system, but using a separate network is better. The user can use a mobile device forgetting about using a physical connection to the system. The RJ45 connector could be used but limits the mobility of the system.\\

The brand of the router used is a Livebox 2. Its features are enough to control and access the main system.\\

\begin{figure}[H]
\begin{centering}
\includegraphics[scale=0.3]{IMGS/LIVE_BOX_2.png}
\caption{Router Live Box 2 \label{Live_Box2}}
\end{centering}
\end{figure}

The router is used to create a communication between the \textit{Raspberry PI 3} and the computer used by the user. Therefore, it is possible to retrieve information from the NoSQL database and manage queries to system.\\

In case it is needed to modify some parameters of the configuration of the system, this network helps the administrator to manage them.\\

The router has a direct connection to the \textit{Raspberry PI 3} via an Ethernet Cable. It was decided to use this connection because it is a reliable media. Using a wireless connection might create problems in the communication, generating latency and jitter to the access of the system and complicating the querying of information in real time.

\subsection{Laptop}

The user will use this device to query information from the system. With the laptop, we can use some tools to retrieve some information from the system. It is useful to use because we can generate some graphics after analysing the information of the database.\\

With the laptop, we can create an \textit{SSH} connection to the system to manage its processes. We can use a \textit{SSH} connection to run processes of the database or to control processes related to the backend. This remote connection allows the administrator of the system to manage it directly.\\

The user can use several operating systems. The \textit{SSH} connection is independent from the operating system. \\

This laptop should include the following software programs:

\begin{itemize}

\item \textit{3T Studio}. This software is used to query the database. Also, it allows to export the information of the database in \textit{JSON}, excel or \textit{CSV}.

\item \textit{Putty or a remote manager for SSH connections}. To start a remote connection to the system.

\item \textit{Microsoft Excel or other similar editors}. To interpret the obtained results of the queries.

\end{itemize}

\section{Technology}

In this part of the document, the technology used in this project is explained. This project looks for the appropriate technology to build the system decreasing its costs associate to hardware components and maintenance.

\subsection{NoSQL database}

In this project, it was decided to use NoSQL technology because of its flexibility. With \textit{NoSQL} databases, we can store all kind of information in a system. When it is necessary to store no structural information, \textit{NoSQL} databases work have a good performance. It is not required to indicate to the system which kind of information is desired to store. NoSQL systems do not mind the structure of the information, they just store it.\\

Other important feature of the \textit{NoSQL} databases is the possibility to store information in real time. In this project, one the goals was to store during long time information. This system has an amount of humidity sensors \cite{humidity_system}, and those sensors measure information about the materials during long periods of time.\\

Developers can build fast and easy queries. In \textit{NoSQL} systems the information is stored in documents, instead of using tables or table spaces.\\

The administrator of a \textit{NoSQL} database can build operations to optimize the access to the databases too. This kind of operations are used in \textit{Big Data} systems and use reducing operations.

\subsection{NodeJS}

NodeJS \cite{node_js, node_js_introduction} offers a solution to manage information from web services. It supports all operations allowed by web services, such as \textit{REST} processes. NodeJS is used in this project as \textit{backend} for a web service.\\

NodeJS works as a process waiting for incoming requests. When NodeJS receives a petition, it processes it and looks for the information in the database. NodeJS works on an asynchronous mode, increasing the performance of a backend. On the other hand, debugging the execution flow of a backed in a hard task to do. If we want to control operations that should work on a synchronous way \textit{callbacks} are required. Callbacks turn the execution flow of NodeJS to a synchronous mode.\\

It has a better performance than \textit{PHP} technologies, which are commonly used by web services. The asynchronous mode creates threads that can serve concurrent petitions. This is one of the reasons to deploy this technology.\\

Performance is one of the goals of this project. The microcontrollers used in this study are limited because of their architecture and \textit{NodeJS} increases the performance of the system.

\subsection{Python 2.7.13}

\textit{Python} is one of the most important components in this project. It is used to control the information which comes from the network of the sensors. Part of the \textit{backend} is developed using \textit{Python}. This part of the backend is a simple program which receives information coming from the \textit{Arduino}. The \textit{Arduino Mega} is directly connected to the \textit{Raspberry PI 3} using an \textit{USB} cable creating a serial communication. It was decided to use this protocol because it is a reliable media used by microcontrollers based on embedded systems.\\

The \textit{Raspberry PI 3} has a process developed in \textit{Python} which is listening to the information coming from the \textit{Serial Port}. \textit{Python} has some libraries to control \textit{Serial} connections. The \textit{Python} program controls the opening, reading and closing operations on the \textit{Serial Port} with a customized \textbf{API} \cite{API} (\textit{Application Programming Interface}) we have developed. The \textit{Python} program uses some libraries to store the information transmitted by the \textit{Arduino Mega}. When the \textit{Python} program detects new information in the \textit{Serial Port} it checks the information received and if this information is valid, it stores it in a \textit{NoSQL} database. If the information received is not valid, the program discards the information received. The information sent by the \textit{Arduino Mega} has an special structure. The Python program knows the structure of the messages and it analyzes the messages trying to parse the information received.\\

The information sent by the \textit{Arduino Mega} has the structure \textit{\textbf{sensor-id \# temperature \# humidity \# date}}. As it is possible to see, the information is delimited by the character \textbf{\textit{\#}}.\\

This program is allocated into the \textit{rc.local} file in the \textit{Raspberry PI 3}. This file allows the \textit{Raspberry PI 3} to execute the script when it is turned on. If the script is not specified on the \textit{rc.local} file we should run it from the Raspberry PI 3 console. We can use the \textit{Raspberry PI 3} console via \textit{SSH} to execute scripts and programs.\\

It was decided to use \textit{Python 2.7.13} because it is one of the most simplest program languages and because it is efficient and reliable.

\subsubsection{API MongoDB}

As it was explained before, To control the information allocated in the \textit{MongoDB's database} is required. \textit{Python} has some libraries to manage MongoDB's databases. This project uses \textit{pymongo} library. This library allows to create queries to manage the database. It is possible to do all operations allowed by databases with this library.\\

The uses this library to inset and query information of the database. The API's webpage brings support to the developer to manage all operations of MongoDB's databases. The documentation of this API is complete and well explained. Also, it supports to install MongoDB systems in Windows, Linux and MacOS.\\

The API manages databases as JSON files. It is not like in other databases systems, where the information is allocated in table spaces. In MongoDB the information is stored in a document. This document appends information to its ending when is required to store structures or variables. The JSON file allows the database administrator to create operations to optimize the access to the database, like Map Reduce operations. It is the most common operation in MongoDB's system.\\

The investigation of how to apply map reduce operations in our databases is not carry out, but it is an open point to consider in future studies. These operations could be done if the database grows fast and the performance of the databases is reduced by the size of the database.

\subsubsection{API Sensor's readings}

This API was developed to control the information sent by the Arduino Mega. This API allows the user to read the information and store it in the MongoDB's database. The API controls reads coming from the serial port. Also, the API controls operations such open and close operations over the MongoDB's database using the API pymongo.\\

This API was developed using Python 2.7.13 and could be used for other systems that want to implement the same methods used in this project.\\

This library checks the information received from the network of the sensors and checks if the structure of the information received is valid. If the received information has a valid structure, it stores the information in the database.\\

This library allows the user of the system to connect \textit{N} humidity sensors. With this feature the user will not have problems to connect more sensors.

\subsection{Scripting}

In other sections of this document we talked about the use of \textit{Python}, \textit{MongoDB} and NodeJS. There is an other important part in this project, scripting. Scripting is important in this project because it allows to automatize the processes used by this system.\\

Normally, the user does realize about this function because it is completely transparent. Scripting allows to control the processes related to the management of the database and the processes which use the network of the sensors. Also, it is possible to manage the information of the database using \textit{NodeJS} as \textit{backend}. It runs queries to retrieve the information of the database\\

In this project was necessary to use \textit{Linux Shell scripting}. \textit{Linux powershell} allows the user to control the processes used by the system.\\

There is a script which controls the processes required to run the measure system on the \textit{Raspberry PI 3}. It checks the status of the system and if there is nothing running it loads all required processes.\\

This method uses the \textit{rc.local} configuration file of the Linux System. This file allows the system to call all functions we want to execute at the boot up.

\section{Methods}

In this section, the methods used in this project are explained. A description of the methodology followed to develop and manage all necessary functions in the system is brought. At least, it is one of the most important topics in this document, because with the methodology described in the following sections, it is possible to obtain the information required to present our results. After that, we can analyze the results and study the conclusions.

\subsection{Querying methods used by NoSQL databases}

This is one the most important topics of this document. We are using a new technology, which uses different methods to obtain the information stored in our database.\\

In NoSQL systems, the queries used to manage databases are different from the others used by a normal SQL system. NoSQL systems does not use data tables. In NoSQL systems, a JSON document is used as a database. The information is not stored like in normal databases systems, which changes the way to obtain and to store data.\\

In this project, there are some steps to follow which are described in the MongoDB's guideline. Here, there are several examples about how to insert, update and query the information of the database.\\

The language used for this purpose has JSON instructions to obtain the information of the database. There are some special keywords to query the information. These special keywords are described in NoSQL documentation too.

\subsection{Methods used to retrieve information from the NoSQL database}

Like it was described before, the database system is completely different to the standard ones. This database system has a different way to retrieve its information.\\

In \textit{MongoDB}, is possible to use some techniques to manage the information of the database. These techniques are the following:

\begin{itemize}

\item Using normal scripting for queries. This is the most difficult way to manage a \textit{NoSQL} database. It is possible to retrieve the information of the database using a Linux console. If it is required to do it, the user has to execute \textit{mongodb} in the Linux terminal and log in with his credentials.\\

On the other hand, there is no assistant to help the user. The only way to obtain information about the commands is typing the option \textit{help} when it is desired to run \textit{mongodb}. A knowledge related to the commands used by \textit{MongoDB} is required.

\item Using a \textit{GUI} applications. There are several \textit{frameworks} which allows the user to manage databases and build queries. In this study it was used \textit{3TStudio} (\textit{MongoCheff} in older versions).\\

This framework has an assistance that allows the user to create his own queries. The user only needs information about the location of the database and the credentials used by the database.

\end{itemize}

The second option is recommended. Using this option makes more efficient to manage a database with an assistant. The user can build its own queries and check if it is required, the syntax of them. It is possible to see the information of the database in real time too.\\

The fist option I think is more complicated to use because the user needs to be experimented with the MongoDB console.

\subsection{Methods used to obtain graphs with the information of the materials}

When the information is stored in the MongoDB's database, if we want to make an analysis of the data that is stored, it is required to export the information.\\

If we want to export the information of the database, we can use 3TStudio to retrieve the information we want to export. It is possible to retrieve the information related to the obtained results after the execution of a MongoDB query.\\

3TStudio allows to select the format to export the information. The results are exported using a \textit{.csv} format. This format is useful if it is required to do an analysis using Microsoft tools. We decided to use Microsoft Excel to analyze the information of the database.\\

%%SEGUIR POR AQUI

Microsoft Excel allows the user to generate graphs and making some calculates of the information that we want to analyze. With Microsoft Excel is easy to obtain the results in graphs. The user only has to select the information to analyze and after that, select the graph desired to interpret the information.\\

In this study, we made two analysis. The analysis made in this study is the following:

\begin{itemize}

\item Analysis of the evolution of the floor. With this analysis we look for information related with the drying process of the floor. This information is displayed in 4 graphs using the values registered from the sensors used in this study.\\

The purpose to select this graph is to analyze if the floor produces valid values in the drying process.

\item Analysis of the results retrieved by the sensors. This analysis allows the interpreter to get information about the current state of the sensors. The sensors show the values along the time.\\

In this analysis it was selected the information registered by the 4 sensors used in this study.

\end{itemize}

\subsection{Methods based on compaction}

\subsection{Methods used to calibrate sensors}

This section explains how is the process selected to calibrate the measure of the sensors. This process, registered the values obtained with the sensors at different conditions. These conditions are the following:

\begin{itemize}

\item Measures obtained after allocating the sensors in water. This method registeres the lowest value measured by the sensors. The lowest value links to the lowest voltage.

\item Measure obtained after allocating the sensors in a dry place. This method registeres the highest value measured by the sensors. The highest value links to the highest voltage.

\item Measure obtained after allocating the sensors in different floors with different values of humidity. This method measures diverse values of floors in different conditions.\\

When the measures of the soils are done, an oven to heat the soils is required. This technique allows to obtain the precise value of humidity of the different floors analyzed. This process takes a day to produce the expected values.

\end{itemize}

When the results are produced, the information, which refers to the measures, is linked to the time inverted to dry a specific soil. Also, the percentage of humidity registered when the drying process is finished refers to the value measured by the sensors before drying the soil. These results define how is the behavior of the sensors when they are measuring a specific material. The chosen process allows to know if the selected sensors used to measure the soils are appropriate.\\

After using the 3 techniques described before, the results produced by the sensors are compared to the values registered by the oven used to dry the materials. With this results, a comparison is done to obtain an average of the results. This average is used to produce precise measures of soils.

\section{Resources}

In this section, the resources used to biuld the system used in this study are explained. The resources are sorted out into two big groups, hardware components and software components. Those two groups are relevant to this project, because with them it is possible to make this study with its results.

\subsection{Hardware}

This group contains the components which form part of the infrastructure of the Information System. The infrastructure is relevant, because it allows to build the Information System and makes easier to monitor the information registered by the sensors. It allows to create a data store to manage the information stored on it.

\subsubsection{Microcontrollers}

The microcontrollers are the intelligent part of the system. They have features to manage the information retrieved by the sensors and store it into the database. In this project, it was decided to use two microcontrollers:

\begin{itemize}

\item \textit{Arduino Mega}. This device allows the system to register the information coming from the network of the sensors. The \textit{Arduino Mega} obtains the information, packages it, and sends it to the \textit{Raspberry PI 3} using serial communication. The \textit{Arduino Mega} is not an inteligent hardware, it obtains the information and sends it to other microcontroller. It has no complex processing.\\

\item \textit{Raspberry PI 3}. This device is the brain of the system. It gets the information which comes from the \textit{Arduino Mega}. When it receives the information, it analyzes it, it checks if the information is well formed, and after that, it stores the correct information into the database. This processing is managed by the backend running a Python program.\\

It has the database too. The database managed by the \textit{Raspberry PI 3} is a \textit{NoSQL} database. The \textit{Raspberry PI 3} has installed \textit{MongoDB} which controls the database.\\

The \textit{Raspberry PI 3} has another backend that is used to manage the petitions of the users. The users want to query information and the \textit{Raspberry PI 3} manages their petitions using a \textit{NodeJS} backend. The \textit{NodeJS} backend queries the information requested by the users and sends it, giving to the user the values stored in the database.

\end{itemize}

\subsubsection{Sensors}

These are one of the most important parts of this project. The sensors are able to get all the information related to the materials which are monitored. The sensors have some electronic features that allow them to monitor soils.\\

In this project, the sensors used for the study are simple and have a low cost. The price of these sensors is around 0.80\euro . and 1\euro . It is possible obtain them via internet. The model used to get the information of the system is \textit{FC-28} \cite{FC_28}. This model is formed by 2 parts. One of those parts is the device that has the electronics to measure the values of the floor, and the other part, is an electrical component which converts the information retrieved by the sensor in numbers.\\

\begin{figure}[H]
\begin{centering}
\includegraphics[scale=0.8]{IMGS/FC-28.png}
\caption{FC-28 humidity sensor \label{FC28}}
\end{centering}
\end{figure}

The information treated by the rectifier uses a numerical structure. This structured has relation with the voltage detected by the sensors. These numerical values has a range between 1024 and 0. The value 1024 corresponds to the driest value measured. The value 0 is the lowest, corresponding to the wettest media analyzed.\\

The connection used is simple. The device which gets the information of the media is connected directly to the rectifier using two cables. The rectifier is connected to the \textit{Arduino Mega} using three cables. The cables used to measure the materials are the following:

\begin{itemize}

\item Positive cable. Provides the positive voltage of the power supply.
\item Ground cable. Provides the negative voltage of the power supply.
\item Data cable. Used to read the values of the measured material.

\end{itemize}

In this project, it was decided to use four sensors with these conditions to build the prototype. The system supports up to twelve sensors in case it is required to measure more materials.

\subsubsection{Network devices}

This section is dedicated to explain the network components which form part of the system. These components provide communication to all the devices which are connected to the system.\\ 

These network devices provide connectivity to the system to query and manage the information stored on it. The network components used in this project are described in the following points:\\

\begin{itemize}

\item \textit{Ethernet cable}. This component is used to communicate the \textit{Raspberry PI 3} with the router of the system. The model of the cable used in this project is a Ethernet cable category \textit{5E}. We decided to use this connection to connect the \textit{Raspberry PI 3} to the router because it is a stable media. Another option is to use a \textit{Wi-Fi} adapter, but \textit{WiFi} communication it is not reliable and it can fail if there are interferences in the media.

\item \textit{Router Livebox 2}. This router is used to create a network between the \textit{Raspberry PI 3} and the users to read information of the \textit{Raspberry PI 3}. This router allows the user to establish a wireless communication too. The user can use a mobile device to connect to the system.

\item \textit{USB cable}. This media is used to communicate the \textit{Arduino Mega} with the \textit{Raspberry PI 3}. We decided to use this communication because it is reliable, stable and easy to manage.

\end{itemize}

The main feature of these components is their reliability, stability and their easy way to manage. It is easy to replace them if it is required. They are common in systems and it is easy to find them.

\subsection{Software}

This section is dedicated to the Software components which form part of the Information System. The software components are the intelligent of the designed system. They allow the user to manipulate the information monitored by the network of the sensors.\\

They bring to the user useful features. These features are the value of the project. All the functions developed follow the requirements defined by the users of the system.\\

Upgrades are supported by the Information System, the software components allow the developer to design new solutions based on the initial state of this system.\\

Also, developers can improve the system, because the system is scalable. The technology selected for this study provides this feature.\\

One of the epics in Sofware Engineering is the possibility to design customized and reusable systems.

\subsubsection{3T Studio}

\textit{3T Studio} is the software used as assistant of queries for the \textit{MongoDB's} database. This tool is available in the Internet and it is possible to download it for free purposes. This tool is available for \textit{Windows}, \textit{Linux} and \textit{MacOs} operative systems.\\

This tool allows the user to manage the information of the database. The user can build queries to retrieve the desire information of the database. The information is displayed into several options, providing to the user the possibility to show the results with lists of components, \textit{JSON} files or displaying the information in tables (e.g. \textit{SQL} systems).\\

This tool has to be connected to a \textit{MongoDB} system. It is required to give the program all the information related to the connection used by the database. The connection to the database is configurable and there are several options to create it:

\begin{enumerate}

\item Over \textit{SSH} tunnel. Based on \textit{SSH} protocol. Provides a higher level of security to the connection in case it is required direct communication to the database.
\item Direct communication to the database. Simplest way to connect to the dabase but sacrifices security concepts.
\item Communication via proxy authentication. Complex but adds a new layer to the security of the system. It is used in huge systems where a higher level of security is required.

\end{enumerate}


\textit{3T Studio} has an assistant to configure the desired connection to the database. In our use cases, the direct communication to the \textit{Database Manager} is the chosen option bacause of its easy way to communicate to the database.\\

On the other hand, the user can export the information of the MongoDB database in several formats. It is possible to use \textit{JSON}, \textit{CSVs} and excel formats. This project uses \textit{CSV} format because it is the easiest way to automate and manage it with Microsoft Excel.

\begin{figure}[H]
\begin{centering}
\includegraphics[scale=0.4]{IMGS/3T_STUDIO.png}
\caption{3T Studio \label{3T_STUDIO}}
\end{centering}
\end{figure}

\subsubsection{Microsoft Excel}

\textit{Microsoft Excel} is one of the tools used to display the results measured by the system. \textit{Microsoft Excel} allows to load \textit{CSV} files exported from the \textit{NoSQL} database.\\

\textit{Microsoft Excel} provides features to apply formulas to the monitored soils. The theoretical values of the measured materials are loaded. The formulas arrange the results obtained with the oven used to calibrate the sensors.\\

The calculates generate an average of the exported information. To analyze the behavior of the soils it is required to select the desired results. The timestamp of the sensor provides a filter to sort the results of the report.\\

When the information is selected, it is required to select the desire chart to represent the results. The generated chart allows the user to compare the produces results with other studies or ideal models.\\

The graphs provided by \textit{Microsoft Excel} allow us to determine if the study is congruent.

\subsubsection{PyCharm}

PyCharm is a development framework which allows the developer to write \textit{Python} scripts. It was decided to use PyCharm because of its special features to develop \textit{Python} applications. Also, it manages the required packages \cite{install_python_modules, pyserial_package, pip} imported by the generated scripts.

\textit{PyCharm} is available over the Internet and it is a free framework to use. It is a stark software and it fits to the needs of the developer. The most important features of this framework are listed as it follows:

\begin{figure}[H]
\begin{centering}
\includegraphics[scale=0.15]{IMGS/PYCHARM.jpg}
\caption{PyCharm Graphycal Interface \label{PYCHARM}}
\end{centering}
\end{figure}

\begin{itemize}

\item \textbf{\textit{Supports several Python's interpreters}}. \textit{PyCharm} has support to use an amount of interpreters. It supports \textit{IronPython} and official \textit{Python}.

\item \textbf{\textit{It supports all Python's versions}}. Using different versions of Python is not risk. It allows the developer to load all the required libraries. It has an assistant to search libraries too.

\item \textbf{\textit{It has a good assistant for development}}. It has a powerful assistant. The assistant has a fair performance when it is required to display the documentation of the methods and libraries imported by the developer.

\item \textbf{\textit{Shortcuts and development from scratch}}. The time invested to develop applications with python is low. Shortcuts and templates are available for the developer.

\item \textbf{\textit{Community version}}. It provides a free to use version with useful tools. Also, there is a professional version which provides more features.

\item \textbf{\textit{Multiplatform}}. \textit{PyCharm} is compatible with Linux, Windows and MacOS operative systems.

\end{itemize}

\subsubsection{Arduino IDE}

Arduino IDE is the framework use to upload code to Arduino Boards. This tool is provided by the manufacturer, \textit{Arduino}.\\

\textit{Arduino} lets the developer to control all kind of sensors. This IDE allows the user to install the controllers needed to work with all available Arduino's boards. It has an assistant to install all the required libraries and drivers for boards.\\

The development with this \textit{IDE} from scratch is possible, it is not required wide development skills. The programs are written in \textit{Arduino} using \textit{C} language. \textit{C} is on a low level layer where the developer controls the functions of the sensors.\\

This \textit{IDE} is available for all operative systems. The framework is uploaded to the official site of  \textit{Arduino's} webpage and it is free to use.\\

\begin{figure}[H]
\begin{centering}
\includegraphics[scale=0.7]{IMGS/ARDUINO_IDE.png}
\caption{Arduino IDE \label{ARDUINO_IDE}}
\end{centering}
\end{figure}

\textit{Arduino} provides to the developer simple instructions and tools to debug programs. It also has a serial console to debug the information measured by the sensors.

\subsubsection{Notepad++}

This tool is not mandatory for development but it is recommended. \textit{Notepad++} lets the developer to program in all kind of programming languages to create software components.

There are other tools to replace this software such as \textit{Sublime}, \textit{Geany} or other open source applications. It was decided to use this program because of its simplicity. Also, there is an initial knowledge about this tool.\\

\textit{Notepad++} was used to develop the subsystem related to the \textit{NodeJS} \textit{backend}.

\subsubsection{Filezilla}

\textit{Filezilla} is a program which allows the user to transfer files between two different devices using file transfer protocols. \textit{Filezilla} was used to transfer all the software components to the \textit{Raspberry PI 3} instead of typing instructions from the command line.\\

\begin{figure}[H]
\begin{centering}
\includegraphics[scale=0.9]{IMGS/FILEZILLA.png}
\caption{Filezilla's Graphycal Interface \label{FILEZILLA}}
\end{centering}
\end{figure}

As it was mentioned before, there are other tools which provide similar features. It was decided to work with \textit{Filezilla} because it is multi-platform.\\

\textit{Filezilla} allows the user to connect to devices using network protocols like \textit{FTP} or \textit{SFTP}. All the required transfers were done over \textit{SFTP} protocol. This protocol requests authentication.\\

With \textit{Filezilla} the user has access to the main file system of the connected device. The access levels are defined by the permissions of the users.

\subsubsection{Putty}

This software is used to manage remote connections to hosts. This program is required for \textit{Windows} systems. Windows does not support remote connections to hosts using the \textit{cmd} terminal.\\

With \textit{Putty}, a \textit{Windows'} user can create a \textit{SSH} or a \textit{Telnet} connection to a remote host. It also supports serial communications. All the remote connections to the Raspberry PI are established via \textit{SSH} communication. This protocol is reliable and secure.\\

\begin{figure}[H]
\begin{centering}
\includegraphics[scale=0.7]{IMGS/PUTTY.PNG}
\caption{Putty Software \label{PYCHARM}}
\end{centering}
\end{figure}

\textit{Putty} is available from its main webpage. It is a free program and it is simple to use.\\

\textit{MacOS} and \textit{Linux} operative systems do not need this software. These platforms support remote connections from the command line.

\newpage
\newpage
