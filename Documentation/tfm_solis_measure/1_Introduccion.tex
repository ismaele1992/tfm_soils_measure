%% Los cap'itulos inician con \chapter{T'itulo}, estos aparecen numerados y
%% se incluyen en el 'indice general.
%%
%% Recuerda que aqu'i ya puedes escribir acentos como: 'a, 'e, 'i, etc.
%% La letra n con tilde es: 'n.

\chapter{Introduction}
\newpage

%\vspace*{1cm}
%párrafo introductorio...

This work studies the measuring reliability of an \textit{Information System} based on low cost sensors, using \textit{free Hardware with Arduinos} \cite{arduino_mega, jjtorres_hardware_2014} boards and ARM-based devices \cite{arm, arm_architecture}.\\

\section{Motivations}

The main reason of the study is to apply Informatics Engineering solutions to innovate on other engineering fields. Moreover, it offers an opportunity to experiment with different technologies in real test environments. Those innovations may offer advantages such as the optimization of processes via automation. \\ 

In our case, this advantage is accomplished via a fast interpretation of results, avoiding mathematical calculations and deprecated tools. This leads to the minimization of the time needed to obtain results. Thus, speeding up breakthroughs in the state-of-the-art. \\

\section{Objectives}

The main objectives of our work are enumerated in this Section:

\begin{enumerate}

\item \textbf{\textit{Apply Informatics Engineering knowledge to solve real life problems}}.

\item \textbf{\textit{Estimate the domain of the problem to be solved}}. 

\item \textbf{\textit{Consider and analyse different technological alternatives}}. Thus, finding an improved solution to a problem. 

\item \textbf{\textit{Design Information Systems useful for the automation of processes}}.

\end{enumerate}

\subsection{Secondary objectives}

The secondary objectives treated in this project are described in the following points:

\begin{enumerate}

\item \textbf{\textit{Use technologies applied in BigData}} \cite{que_es_big_data, importancia_big_data, big_data_applications} e.g. NoSQL \cite{que_es_no_sql, sql_vs_nosql, nosql_databases} databases, which are interesting when dealing with high dimension and non-structured data.

\item \textbf{\textit{Study new database management tools}}. This allows to improve the management of high dimension databases.

\item \textbf{\textit{Apply programming knowledge to sensor networks}}.

\item \textbf{\textit{Manage Unix systems}}.

\end{enumerate}


%El principal objetivo de este proyecto radicará en demostrar la viabilidad de este nuevo método de %invisibilización así como sentar las bases de sus fundamentos. Así pues, para lograr la consecución de %este objetivo, se realizarán una serie de actividades encaminadas al desarrollo del método y su posible %mejora, dichas actividades se puntualizan a continuación:

%\begin{itemize}
    %\item Reutilización de códigos empleados para la optimización de ``\textit{cloaking}''.
    %\item Validación de códigos y corrección de errores.
    %\item Generación y análisis de resultados.
    %\item Valoración de la viabilidad del método.\\

%\end{itemize}

%Al mismo tiempo se persiguen otros objetivos secundarios, enumerados a continuación, que persiguen mejorar la usabilidad y la evaluación del método:

%\begin{itemize}

    %\item Optimizaciones de eficiencia.
    %\item Implementación de herramientas complementarias.\\
%\end{itemize}

%Por último se fijó un pequeño objetivo paralelo al proyecto y encaminado al aprendizaje del uso de una %potente herramienta, aprender lenguaje \LaTeX{} para la redacción de la memoria.

\section{Organization of the memory}

This document comprises 8 chapters and a bibliography section.\\

Chapter 1 offers and introduction to the selected study and the criteria for its design and components.\\

Chapter 2 presents the state-of-the-art.\\

Chapter 3 describes what methods are available to gather requirements according to the stakeholders' needs and presents which method was chosen for this project.\\

The design of the \textit{Information System} \cite{information_system, information_system_architecture} is described in Chapter 4.\\

The materials and methods used for this work are addressed in Chapter 5.\\

The results of the study are shown and analysed in Chapter 6.\\

Finally, Chapter 7 enumerates the conclusions of the study and presents its future work.\\

\newpage
\newpage
