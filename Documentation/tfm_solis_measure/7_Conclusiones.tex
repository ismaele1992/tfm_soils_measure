%% Los cap'itulos inician con \chapter{T'itulo}, estos aparecen numerados y
%% se incluyen en el 'indice general.
%%
%% Recuerda que aqu'i ya puedes escribir acentos como: 'a, 'e, 'i, etc.
%% La letra n con tilde es: 'n.

\chapter{Conclusions and future lines}
\newpage

This chapter is dedicated to present the different conclusions of the study. Also, there is an explanation of the different future lines of this study which are opened to improve the developed system.

\section{Conclusions}

In this section the main conclusions of our study are presented. 

\begin{enumerate}

\item \textit{The price of a sensor does not determine its validity for the project}. It was proven that using a low cost sensor does not imply that the information obtained cannot be valid and useful.\\

Using statistical analysis tools allows us to obtain acceptable results.

\item \textit{The use of new technologies in the field of Computer Engineering are completely applicable to other Engineering fields}.\\

Using NoSQL databases and free hardware provides us solutions that can be applied to other engineering areas. With NoSQL databases we can monitor huge amounts of information that can be analysed.\\

Using free hardware allows the creation of Information Systems that can be adjusted to other user needs.

\item \textit{Flexibility is quality too}. Developing an Information System like the system that is referenced in this document, allows the user to realize that his needs can be covered and also, the user can adapt his future needs to an Information System that it is continuously evolving. The quality of a product is determined by the final user.

\item \textit{The requirements of the users are the Computer Engineer's business}. Bringing informatics solutions to users with requirements that are evolving continuously is the real market niche for all Informatics Engineers. Requirements do not have to remain static during all times. It is normal that the bigger projects are developed when the users have requirements that are continuously evolving.

\end{enumerate}


%En primer lugar, se constata el hecho de que al incidir con ondas planas en placas con las características de las analizadas se forma una onda estacionaria. A pesar de ser muestras que no presentan una homogeneidad en ellas es capaz de producirse una onda de tales características con puntos con un factor de realce muy elevados.\\

%Para continuar habría que comentar como son estas ondas y más en concreto sus máximos. Como se menciona en el capítulo 2, aprovechándose de la técnica denominada como SERS se podría haber llegado a un orden de magnitud de orden $10^8$. Si repasamos este capítulo podemos encontrar que ese orden de magnitud, y la aproximación que se utiliza para determinar ese resultado sería de gran exactitud para valores de longitudes de ondas cercanas a los azules. En nuestro caso usamos una longitud de onda de $655nm$ la cual se encuentra entre los rojos, zona del espectro visible para la cual no es bueno usar la aproximación de $|E|^4$. Por lo tanto el factor de realce disminuye en gran medida y es por ello que estamos en ordenes de $10^4$.

\section{Future lines}

After the analysis of the conclusions, the possible future lines are presented. 

\begin{enumerate}

\item \textbf{\textit{Increasing the computational resources of the \textit{Information System} that has been designed}}. Like every \textit{Information System}, it is always possible to obtain a higher number or computational resources. In this work, it was developed an \textit{Information System} with a computational core based on \textit{Raspberry PI 3}. The \textit{Raspberry PI 3} is a simple processing module with limited specifications. If needed, it is possible to upgrade the system with a more powerful hardware. Some of the specifications that could be improved are the following.\\

\begin{itemize}

\item \textit{A more powerful processor}. The processor of the Raspberry PI 3 has four cores based on an ARM architecture. ARM architecture is characterized by processors suitable for mobile devices with low energy consumption. It is recommended to use a device with a processor with an architecture based on 8086 processors. These processors are more powerful and can do more complex calculations.

\item \textit{More RAM memory}. The device used in the prototype has 1GB of RAM memory. Despite its RAM memory the device is capable of dealing with low volumes of information. It is able to query and retrieve information from the used database, storing ambient information. The problem might arise in the future when the data will grow to higher volumes of information (like it happens in systems used in \textit{BigData}) and the system would have slower response than expected. To avoid this problem, it would be necessary to increase the RAM memory.

\end{itemize}

\item \textbf{\textit{Data redundancy}}. This requirement was not deployed in this prototype. It would be necessary to create a redundancy in the database with the information of the ambient variables to avoid a possible loss of information if the devices used for storage fail. to do so, it is viable to use a storage system that allows the replication of information in the database.

\item \textbf{\textit{Replace sensors}}. In this prototype low cost sensors that have a limited lifetime were used. To avoid the constant replacement of the sensors it would be necessary to use sensors with better quality. These sensors will have a longer lifetime than the sensors used in this prototype.

\item \textbf{\textit{Using Map Reduce algorithms}}. The use of these algorithms could optimize the time spent by queries when it is needed to retrieve the information stored in the system. These algorithms are very common in \textit{BigData}.

\item \textbf{\textit{Development of the Front End}}. The font end of the graphical interface was not developed in this study. Consider this feature is reachable future line.

\item \textbf{\textit{Improve web back-end}}. The back-end of this project is developed but with limited features. Considering this feature will open new features for the future system.

\end{enumerate}




\newpage